\documentclass[a4paper,10pt]{article}
%\documentclass[a4paper,10pt]{scrartcl}



\usepackage[utf8]{inputenc}
\usepackage{fancyhdr}
\usepackage{graphicx}
\usepackage[top=1in, bottom=1in, left=1in, right=1in]{geometry}

\title{Vereinssatzung FFHB}
\author{Jelto Wodstrcil}
\date{22.11.14}

\pdfinfo{%
  /Title    (Vereinssatzung FFHB)
  /Author   (Jelto Wodstrcil)
  /Creator  (Jelto Wodstrcil)
  /Producer (Jelto Wodstrcil)
  /Subject  (Satzung)
  /Keywords ()
}

\renewcommand{\thesection}{\S\arabic{section}}
\graphicspath{ {/} }
\pagestyle{fancy}
\fancyhf{}
\lhead{\includegraphics[width=1cm]{logo}}
\rhead{Vereinssatzung Freifunk Bremen}
\rfoot{Seite \thepage}


\begin{document}
\maketitle

\begin{figure}[!]
 \includegraphics[width=5cm]{logo}
 \centering
\end{figure}

\section{Name und Sitz des Vereins, Geschäftsjahr}
  \begin{itemize}
    \item Der Verein führt den Namen "Freifunk Bremen". Er soll in das Vereinsregister der Stadt Bremen eingetragen werden und danach den Zusatz "e.V." führen. Im folgenden Verein genannt.
    \item Der Verein hat seinen Sitz in Bremen.
    \item Das Geschäftsjahr ist das Kalenderjahr.
  \end{itemize}

\section{Zweck des Vereins}
  \begin{itemize}
    \item Zweck des Vereins ist die Erforschung, Verbreitung und Anwendung freier Netzwerktechnologien sowie die Verbreitung, Anwendung und Vermittlung von Wissen über Netzwerke sowie kultureller Inhalte mittels elektronischer Medien. 
    \item Weiterhin fördert der Verein ideell, materiell und/oder finanziell: 
    \begin{itemize}
      \item den Zugang von benachteiligten Personengruppen zur Informationsgesellschaft durch die Bereitstellung eines frei zugänglichen Netzwerk und dem Wissenstransfer zur verantwortungsvollen Nutzung des Netzes;
      \item die Schaffung experimenteller Kommunikations- und Infrastrukturen unabhängiger Bürgernetze 
      \item die Veranstaltung regionaler, nationaler und internationaler Kongresse, Treffen, Konferenzen und Seminare und die Bereitstellung von Know-How über Technik und Anwendung freier Netzwerke (Routingprotokolle, Betriebssysteme, 802.11x, Meshverfahren, VPNs, Webservices und andere moderne Verfahren zum Betrieb einer IT-Infrastruktur);
      \item kulturelle, technologische und soziale Bildungs- und Forschungsprojekte 
      \item die Information über gesellschaftliche, kulturelle, gesundheitliche, rechtliche und weitere Auswirkungen freier Netzwerke;
    \end{itemize}
    \item Eine Änderung des Vereinszweck darf nur im Rahmen des in § 3 (1) gegebenen Rahmens erfolgen. 
  \end{itemize}
  
\section{Gemeinnützigkeit}
  \begin{itemize}
   \item Der Verein verfolgt im Rahmen seiner Tätigkeit gemäß § 2 der Satzung ausschließlich und unmittelbar gemeinnützige Zwecke im Sinne des Abschnittes "steuerbegünstigte Zwecke" der Abgabenordnung (§§ 51 AO). 
   \item Er ist selbstlos tätig und verfolgt nicht in erster Linie eigenwirtschaftliche Zwecke. Es ist nicht vorgesehen, dass Erkenntnisse finanziell verwertet werden. 
   \item Mittel des Vereins dürfen nur für die satzungsmäßigen Zwecke verwendet werden. Die Mitglieder erhalten keine Zuwendungen aus den Mitteln des Vereins.
   \item Die Mitglieder erhalten keine Gewinnanteile. Es darf keine Person durch Ausgaben, die dem Zweck des Vereins fremd sind, oder durch unverhältnismäßig hohe Vergütungen begünstigt werden. Bei Ausscheiden oder Auflösung dürfen Vereinsmitglieder keine Anteile des Vereinsvermögens erhalten.
  \end{itemize}

\section{Mitgliedschaft}
  \begin{itemize}
    \item Mitglieder des Vereins können natürliche und juristische Personen werden, die sich bereit erklären, die Vereinszwecke und - ziele aktiv, finanziell oder materiell zu unterstützen. Bei Minderjährigen ist die Zustimmung des gesetzlichen Vertreters erforderlich.  
    \item der Verein besteht aus ordentlichen Mitgliedern und Fördermitgliedern. Ordentliche Mitglieder sind in der Mitgliederversammlung stimmberechtigt.
    \item Ordentliches Mitglied des Vereins kann jede natürliche Person werden, die sich mit den Zielen des Vereins verbunden fühlt und den Verein aktiv fördern will. Die Mitgliedschaft ist in Textform (§ 126b BGB) auf Papier oder elektronisch zu beantragen. Über den Antrag entscheidet der Vorstand.
    \item Das aufgenommene Mitglied erhält eine Kopie der Satzung. Die jeweils aktuelle Satzung wird darüber hinaus an geeigneter Stelle den Mitgliedern verfügbar gemacht
    \item Fördermitglied des Vereins kann jede natürliche oder juristische Person werden, die sich mit den Zielen des Vereins verbunden fühlt und den Verein finanziell und ideell unterstützen will. Die Mitgliedschaft ist in Textform (§ 126b BGB) auf Papier oder elektronisch zu beantragen. Über den Antrag entscheidet ein Vorstandsmitglied.
    \item Gegen den ablehnenden Bescheid des Vorstands, der mit Gründen zu versehen ist, kann der Antragsteller Beschwerde erheben. Die Beschwerde ist innerehalb eines Monats ab Zugang des ablehnenden Bescheids schriftlich beim Vorstand einzulegen. Über die Beschwerde entscheidet die nächste ordentliche Mitgliederversammlung.
  \end{itemize}
  
\section{Beginn und Ende der Mitgliedschaft}
  \begin{itemize}
   \item Die Mitgliedschaft wird gegenüber dem Vorstand durch schriftliche Erklärung beantragt. Dieser entscheidet innerhalb von vier Wochen über die Aufnahme. Ablehnungen werden in gleicher Frist schriftlich begründet. 
   \item Die Mitgliedschaft endet durch den Tod des Mitglieds oder durch freiwilligen Austritt, bei juristischen Personen auch durch Verlust der Rechtspersönlichkeit. 
   \item Der freiwillige Austritt erfolgt durch gegen über einem Mitglied des Vorstands in Textform. Er ist nur zum Schluss eines Quartals unter Einhaltung einer Kündigungsfrist von zwei Wochen zulässig.
   \item Der Vereinsausschluss erfolgt durch Beschluss des Vorstandes oder Mehrheitsbeschluss der Mitgliederversammlung, wenn ein Mitglied gegen die Ziele und Interessen des Vereins schwer verstoßen hat oder die Voraussetzungen der Satzung nicht mehr erfüllt. 
   \item Vor der Beschlussfassung ist dem Mitglied unter Setzung einer angemessenen Frist Gelegenheit zu geben, sich persönlich vor dem Vorstand oder in Textform zu rechtfertigen. Eine in Textform vorliegende Stellungnahme des Betroffenen ist in der Vorstandssitzung zu verlesen. Der Beschluss über den Ausschluss ist mit Gründen zu versehen und dem Mitglied in textform (E-Mail oder Brief) bekannt zu machen. Gegen den Ausschließungsbeschluss des Vorstands steht dem Mitglied das Recht der Berufung an die Mitgliederversammlung zu. Die Berufung hat aufschiebende Wirkung. Die Berufung muss innerhalb einer Frist von einem Monat ab Zugang des Ausschließungsbeschlusses beim Vorstand schriftlich eingelegt werden. Ist die Berufung rechtzeitig eingelegt, so hat der Vorstand innerhalb von zwei Monaten die Mitgliederversammlung zur Entscheidung über die Berufung einzuberufen. Geschieht das nicht, gilt der Ausschließungsbeschluss als nicht erlassen. Macht das Mitglied von dem Recht der Berufung gegen den Ausschließungsbeschluss keinen Gebrauch oder versäumt es die Berufungsfrist, so unterwirft es sich damit dem Ausschließungsbeschluss mit der Folge, dass die Mitgliedschaft als beendet gilt.
   \item Bei Beendigung der Mitgliedschaft, gleich aus welchem Grund, erlöschen alle Ansprüche aus dem Mitgliedsverhältnis. Eine Rückgewähr von Beiträgen, Spenden oder sonstigen Unterstützungsleistungen ist grundsätzlich ausgeschlossen. Der Anspruch des Vereins auf rückständige Beitragsforderungen bleibt hiervon unberührt. 
  \end{itemize}

\section{Rechte und Pflichten der Mitglieder}
  \begin{itemize}
   \item Die Mitglieder sind berechtigt, an allen angebotenen Veranstaltungen des Vereins teilzunehmen. Sie haben darüber hinaus das Recht, gegenüber dem Vorstand und der Mitgliederversammlung Anträge zu stellen. 
   \item Der Verein erhebt einen Mitgliedsbeitrag, zu dessen Zahlung die Mitglieder verpfichtet sind. Näheres regelt eine Beitragsordnung, die von der Mitgliederversammlung beschlossen wird.
   \item Mitglieder des Vereins sind verpflichtet eine gültige Emailadresse anzugeben zur Zustellung von Vereinsinformationen.
  \end{itemize}
  
\section{Mitgliedsbeiträge und Finanzierung}
  \begin{itemize}
   \item Die erforderlichen Geld- und Sachmittel des Vereins werden beschafft durch: 
   \begin{itemize}
    \item Mitgliedsbeiträge (§7) 
    \item Spenden
    \item Fördermittel, Zuschüsse und sonstige Zuwendungen. 
   \end{itemize}
   \item Für die Höhe der jährlichen Mitgliederbeiträge, Förderbeiträge, Aufnahmegebühren, Umlagen, ist die jeweils gültige Beitragsordnung maßgebend, die von der Mitgliederversammlung beschlossen wird.
   \item Auf Antrag kann der Vorstand Mitgliedsbeiträge ganz oder teilweise erlassen. 
  \end{itemize}


\section{Organe des Vereins}
  \begin{itemize}
   \item Die Organe des Vereins sind:
   \begin{itemize}
    \item Die Mitgliederversammlung,
    \item Der Vorstand.
   \end{itemize}
  \end{itemize}

\section{Mitgliederversammlung}
  \begin{itemize}
   \item Der Mitgliederversammlung gehören alle ordentlichen Vereinsmitglieder mit je einer Stimme an. 
   \item Die Mitgliederversammlung ist das oberste Beschlussorgan des Vereins. Ihr obliegen alle Entscheidungen, die nicht durch die Satzungen oder die Geschäftsordnung einem anderen Organ übertragen werden.
   \item Beschlüsse werden von der Mitgliederversammlung durch öffentliche Abstimmung getroffen. Auf Wunsch eines ordentlichen Mitglieds ist geheim abzustimmen.
   \item Eine ordentliche Mitgliederversammlung, bezeichnet als Jahreshauptversammlung, wird einmal jährlich einberufen. Sie wird vom Vorstand per Email unter Angabe der Tagesordnung einberufen. Die Einladungsfrist beträgt 4 Wochen. Das Einladungsschreiben gilt dem Mitglied als zugegangen, wenn es an die letzte vom Mitglied des Vereins schriftlich bekannt gegebene Emailadresse gerichtet ist. 
   \item Eine außerordentliche Mitgliederversammlung kann jederzeit einberufen werden, wenn mindestens 25\% der ordentlichen Mitglieder oder der Vorstand dies jeweils gemäß § 12 unter Angabe eines Grunds beantragen. Dem angegebenen Grund mussen die gewünschten Tagesordnungspunkte zu entnehmen sein; sie werden auf die Einladung übernommen.
   \item Die Mitgliederversammlung ist bei ordnungsgemäßer Einladung ohne Rücksicht auf die Anzahl der Erschienenen beschlussfähig. Sie wählt aus ihrer Mitte einen Versammlungsleiter. Beschlüsse werden, sofern die Versammlung nicht etwas anderes bestimmt, mit einfacher Mehrheit getroffen. 
   \item Dem Vorstand obliegt zu allen Mitgliederversammlungen die Festsetzung eines Termins und die rechtzeitige Einladung aller Mitglieder bis spätestens zwei Wochen vor dem von ihm festgelegten Termin. Bei von den Mitgliedern beantragten Mitgliederversammlungen darf der Termin nicht mehr als acht Wochen nach dem Eingang des Antrags beim Vorstand liegen.
   \item Uber die Beschlüsse der Mitgliederversammlung ist ein Protokoll anzufertigen, das vom Versammlungsleiter und vom Schriftführer zu unterzeichnen ist. Das Protokoll ist innerhalb von 14 Tagen allen Mitgliedern zugänglich zu machen und auf der nächsten Mitgliederversammlung genehmigen zu lassen.
   \item Erreicht eine Mitgliederversammlung nicht die Beschlussfähigkeit, ist die darauffolgende ordentlich anberaumte Mitgliederversammlung beschlussfähig, wenn mindestens sieben Mitglieder anwesend sind.
  \end{itemize}
  
\section{Aufgaben der Mitgliederversammlung}
  \begin{itemize}
   \item Oberstes Organ des Vereins ist die Mitgliederversammlung, sie ist grundsätzlich für alle Aufgaben zuständig, sofern bestimmte Aufgaben gemäß dieser Satzung nicht einem anderen Vereinsorgan übertragen wurde. Sie hat insbesondere folgende Aufgaben: 
   \begin{itemize}
    \item Die Jahresberichte entgegenzunehmen und zu beraten, 
    \item Rechnungslegung für das abgelaufene Geschäftsjahr, 
    \item Entlastung des Vorstands, 
    \item (im Wahljahr) den Vorstand zu wählen, 
    \item über die Satzung, Änderungen der Satzung sowie die Auflösung des Vereins zu bestimmen, 
    \item die Kassenprüfer zu wählen, die weder dem Vorstand noch einem vom Vorstand berufenen Gremium angehören und nicht Angestellte des Vereins sein dürfen, 
    \item Entscheidung über Gebührenbefreiungen
    \item Aufgaben des Vereins, 
    \item An- und Verkauf von Eigentum, 
    \item Beteiligung an Projekten nach § 2, 
    \item Entscheidung über die Ablehnung von Spenden und Fördermitteln, 
    \item Genehmigung aller Geschäftsordnungen für den Vereinsbereich, 
    \item Mitgliedsbeiträge. 
   \end{itemize}
   \item Sie kann über weitere Angelegenheiten beschließen, die ihr vom Vorstand oder aus der Mitgliedschaft vorgelegt werden. 
  \end{itemize}
  
\section{Änderung von Satzungs- und Geschäftsordnung}
  \begin{itemize}
   \item Uber Änderungen von Satzungs- und/oder Geschäftsordnung kann in der Mitgliederversammlung nur abgestimmt werden, wenn auf diesen Tagesordnungspunkt hingewiesen wurde und der Einladung sowohl der bisherige als auch der vorgesehene neue Text beigefügt war.
   \item Satzungsänderungen, die von Aufsichts-, Gerichts- oder Finanzbehörden aus formalen Gründen verlangt werden, kann der Vorstand von sich aus vornehmen. Diese Satzungsänderungen müssen der nächsten Mitgliederversammlung mitgeteilt werden.
  \end{itemize}

\section{Auflösung des Vereins und Vermögensbindung}
  \begin{itemize}
   \item Die Auflösung des Vereins muss von der Mitgliederversammlung mit einer Mehrheit von drei Vierteln beschlossen werden. Die Abstimmung ist nur möglich, wenn auf der Einladung zur Mitgliederversammlung als einziger Tagesordnungspunkt die Auflösung des Vereins angekündigt wurde.
   \item Bei Aufösung des Vereins, Aufhebung der Körperschaft oder Wegfall der gemeinnützigen Zwecke darf das Vermögen der Körperschaft nur für steuerbegünstigte Zwecke verwendet werden. Zur Erfüllung dieser Voraussetzung wird das Vermögen einer anderen steuerbegünstigten Körperschaft oder einer Körperschaft öffentlichen Rechts für steuerbegünstigte Zwecke übertragen, die ebenfalls den Auftrag zur Bildung und Volksbildung im Umgang mit Informationstechnologie wahrnimmt. Näheres kann die Geschäftsordnung regeln.
   \item Der Grundsatz der Vermögensbindung ist bei der Fassung von Beschlüssen über die künftige Verwendung des Vereinsvermögens zwingend zu erfüllen.
   \item Bei Verlust der Anerkennung als gemeinnütziger Verein gelten die vorgenannten Absätze analog. Das Vermögen und die Güter des Vereins werden entsprechend übertragen.
   \item Bei der Aufösung oder Aufhebung der Körperschaft fällt das Vermögen der Körperschaft an den "Förderverein Freie Netzwerke e.V." der es unmittelbar und ausschließlich zu gemeinnützigen Zwecken zu verwenden hat.
  \end{itemize}


\section{Vorstand}
  \begin{itemize}
   \item Der Vorstand besteht im sinne des § 26 BGB aus mindestens drei ordentlichen Mitgliedern: dem Vorstandsvorsitzenden, dem Schatzmeister und dem Schriftführer. Des Weiteren können bis zu drei Beisitzer in den Vorstand gewählt werden. Es kann auf Wunsch der Mitgliederversammlung auf eine Wahl der Beisitzer verzichtet werden.
   \item Der Verein kann gerichtlich und außergerichtlich von jedem einzelnem Vorstandsmitglied vertreten werden. Sie werden von der Mitgliederversammlung für die Dauer von vier Jahren gewählt. Die Wiederwahl von Vorstandsmitgliedern ist zulässig. Nach Fristablauf bleiben die Vorstandsmitglieder bis zum Antritt ihrer Nachfolger im Amt. 
   \item Der Vorstand kann sich eine Geschäftsordnung geben und kann besondere Aufgaben unter seinen Mitgliedern verteilen oder Ausschüsse für deren Bearbeitung oder Vorbereitung einsetzen. 
   \item Der Vorstand beschließt über alle Vereinsangelegenheiten, soweit sie nicht eines Beschlusses der Mitgliederversammlung bedürfen. Er führt die Beschlüsse der Mitgliederversammlung aus. 
   \item Der Vorstand tritt auf Verlangen eines Vorstandsmitgliedes nach Absprache mit den anderen Vorstandsmitgliedern zusammen. 
   \item Der Vorstand ist bei Anwesenheit von 3 Mitgliedern beschlussfähig. Er fasst Beschlüsse mit Stimmenmehrheit. Beschlüsse des Vorstands können bei Eilbedürftigkeit auch schriftlich oder fernmündlich gefasst werden, wenn alle Vorstandsmitglieder ihre Zustimmung zu diesem Verfahren schriftlich oder fernmündlich erklären. Schriftlich oder fernmündlich gefasste Vorstandsbeschlüsse sind schriftlich niederzulegen und zu unterzeichnen. 
   \item Über Konten des Vereins können die Vorstandsmitglieder einzeln und selbständig verfügen. 
   \item Satzungsänderungen, die von Aufsichts-, Gerichts- oder Finanzbehörden aus formalen Gründen verlangt werden, kann der Vorstand von sich aus vornehmen. Diese Satzungsänderungen müssen der Mitgliederversammlung mitgeteilt werden. 
   \item Scheidet ein Vorstandsmitglied vor Ablauf seiner Wahlzeit aus, ist der Vorstand berechtigt, ein kommissarisches Vorstandsmitglied zu berufen. Auf diese Weise bestimmte Vorstandsmitglieder bleiben bis zur nächsten Mitgliederversammlung im Amt. Es ist unmittelbar eine außerordentliche Mitgliederversammlung einzuberufen mit dem Zweck der Neuwahl des Vorstands. 
   \item Der Vorstand ist berechtigt, während der Gründungsphase redaktionelle Änderungen in der Satzung mit rein formalen Charakter selbstständig zu tätigen. 
   \item Der Schatzmeister überwacht die Haushaltsführung und verwaltet das Vermögen des Vereins.
   \item Vorstandsmitglieder können jederzeit von ihrem Amt zurücktreten.
   \item Die Vorstandsmitglieder sind grundsätzlich ehrenamtlich tätig. Sie haben Anspruch auf Erstattung notwendiger Auslagen, deren Rahmen von der Geschäftsordnung festgelegt wird.
   \item Jedes Vorstandsmitglied hat bei Abstimmungen des Vorstands eine Stimme. Bei Abstimmungen ist eine Mehrheit von zwei Dritteln der abgegebenen gültigen Stimmen nötig.
  \end{itemize}

\section{Inkrafttreten}
  \begin{itemize}
   \item Vorstehender Satzungsinhalt wurde von der (x Gegenstimmen, x Enthaltungen) Gründungsversammlung am x.x.2014 einstimmig beschlossen. Diese Satzung tritt mit der Eintragung in das Vereinsregister in Kraft. 
   \item Außerdem wurde dieser Text (erstmalig) in der Mitgliederversammlung am x.x.2014 überarbeitet und die Änderungen einstimmig von allen anwesenden sowie stimmberechtigten Mitgliedern beschlossen. 
  \end{itemize}

  
\end{document}


