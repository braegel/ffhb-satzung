\documentclass[a4paper,10pt]{article}
%\documentclass[a4paper,10pt]{scrartcl}

\usepackage[utf8]{inputenc}

\title{Beitragsordnung Freifunk Bremen e.V.}
\author{Jelto Wodstrcil}
\date{16.01.2015}

\pdfinfo{%
  /Title    (Beitragsordnung Freifunk Bremen e.V.)
  /Author   (Jelto Wodstrcil)
  /Creator  ()
  /Producer ()
  /Subject  ()
  /Keywords ()
}

\renewcommand{\thesection}{\S\arabic{section}}

\begin{document}
\maketitle

\section{Mitgliedsbeiträge}
\begin{itemize}
 \item Jedes Mitglied hat einen Mitgliedsbeitrag zu entrichten.
 \item Der Mitgliedsbeitrag beträgt für ordentliche Mitglieder jährlich 60,00 Euro.
 \item Mitglieder, die einen höheren Beitrag zahlen, erwerben mit Zahlung den Status "Förderndes Mitglied".
\end{itemize}

\section{Ermäßigungen}
\begin{itemize}
 \item Jedes Mitglied hat das Recht einen Antrag auf verminderten Mitgliedsbeitrag in Höhe von 12 Euro zu stellen. Der Vorstand entscheidet über jeden Antrag im Einzelfall. 
\end{itemize}

\section{Fälligkeit und Zahlungsweise}
\begin{itemize}
 \item Die Mitgliedsbeiträge werden jeweils zum 01. März im laufenden Geschäftsjahr fällig.
 \item Die Zahlung ist per Überweisung des Betrags auf das Vereinskonto zu zahlen. 
 \item In Ausnahmefällen kann auch eine Barzahlung an den Schatzmeister geleistet werden. Dies ist jedoch vorab mit dem Schatzmeister zu vereinbahren. 
 \item Der Mitgliedsbeitrag für das laufende Jahr wird anteilig für jeden noch nicht angefangenen Monat berechnet. 
 \end{itemize}

\section{Aufnahmegebühren und Starterpaket}
\begin{itemize}
 \item Aufnahmegebühren fallen nicht an.
\end{itemize}

\section{Mahnwesen und Inkasso}
\begin{itemize}
 \item Mitglieder, die mit der Zahlung ihres Beitrages mehr als einen Monat in Rückstand sind, sind zu mahnen. Bleibt die Mahnung erfolglos, ist sie nach einem weiteren Monat zu wiederholen Bleibt diese Mahnung erfolglos, kann der Schatzmeister den Ausschluss des Mitglieds beantragen. 
\end{itemize}

\section{Pflichtdienste}
\begin{itemize}
 \item Pflichtdienste gibt es nicht.
\end{itemize}

\section{Gründungsklausel}
\begin{itemize}
 \item Der Verein soll in das Vereinsregister eingetragen werden. Falls für die Eintragung in das Vereinsregister oder für die Anerkennung der Gemeinnützigkeit durch die entsprechenden Behörden Änderungen oder Anpassungen der Satzung nötig werden, kann der Vorstand diese auch ohne Beschluss der Mitgliederversammlung vornehmen. Der Vorstand wird zur Vornahme dieser Handlungen insoweit bereits jetzt ausdrücklich ermächtigt. 
\end{itemize}


\section{Inkrafttreten und Geltungsdauer}
\begin{itemize}
 \item Diese Finanzordnung gilt zeitlich unbegrenzt und kann nur durch die Mitgliederversammlung geändert werden. 
 \item Redaktionelle Änderungen sind hiervon nicht betroffen.
\end{itemize}



\end{document}
